\documentclass[a4paper, 11pt, titlepage]{article}
\usepackage{a4wide}
\usepackage{amsmath}

\usepackage[pdftex]{graphicx}
\usepackage{pdfpages}
\usepackage{subcaption}
\usepackage{listings}
\usepackage{color}
\usepackage{cite}
\usepackage{import}
\usepackage[title,titletoc,toc]{appendix}


\newcommand{\HRule}{\rule{\linewidth}{0.5mm}}

%Colours!
\usepackage[table]{xcolor}
\definecolor{mygreen}{rgb}{0,0.6,0}
\definecolor{mygray}{rgb}{0.5,0.5,0.5}
\definecolor{mymauve}{rgb}{0.58,0,0.82}
\definecolor{mynavyblue}{HTML}{1E5A9C}

%PDF hyperlinks
\usepackage{hyperref}
\hypersetup{
     colorlinks   = true,
     citecolor    = black,
     linkcolor    = black,
     urlcolor   = mynavyblue,
     bookmarksopen  = false,
     pdfpagemode  = UseNone,
     pdftitle   = {Hexacopter Project 2015 - Ardupilot proposal},
     pdfauthor    = {},
     pdfsubject = {Hexacopter FYP 2015}
}



\begin{document}


  \input{./title.tex}

  %\maketitle
  \begin{abstract}
Abstract

  \end{abstract}

  \tableofcontents

  \section{Introduction}
	Background, Aims

\section{History / prior works / relevance}
	Chase-cam (Lily cam, ...Dog)
	Cinematography	feature films, Red Bull, 
	Remote Inspection, approached by mining industry technology start-ups

\section{Motivation}
	I really really want one.
	computer vision assisted control routine
	Obstacle avoidance
	Introduction to SLAM systems

\section{Project log}

	Platform
		Description
			Frame, Flight controller, processor, gimbal
		Platform Weaknesses
			(fitness for purpose etc)
			NAZA not hackable, deliberately obfuscated comms
			RC switch did not permit altitude control
			No telemetry
			Waypoints handled on the embedded system despite GPS loiter behaviour of the flight computer.


	Team Achievements
		Conversion to Ardupilot 3.3
		Mavlink, waypoints, altidude control, failsafes, telemetry
		Full, and revocable hand-over to external autopilot.
		Linearised SI units input and output from flight controller
		Reduction of mass, tidy wiring harness
		
		Web UI and HUD
		Waypoint modifier using no-fly zones (not integrated with ardupilot exclusion zones)
		Glyph detection
		Capture and tagging of images for proprietary offline 3D reconstruction.

\section{Object tracker Theory}
	Position estimation
	Desired relative pose

	Observations and uncertainties
		Structure from image
			Computational load
			Live capture
		Uncertainties:
			A good uncertainty model encodes what is known and what is unknown.
			Good treatment of uncertainties combines knowledge without losing information, or adding assumptions.
			In the case of structure from image, a system that extracts maximal information from a single camera should be automatically capable of full stereoscopy using only the uncertainty analyses that applied to the monocular case.
				Monocular Camera uncertainty model
				Lidar Lite Uncertainty model
				GPS uncertainty model
				Time evolution uncertainty models
				Base Utilities
					Vector sum
					combination
					point sample
				Kalman style filtering
			Relative uncertainties
				Networks of relative measurements
				Graph traversal
					Nodes:
						Objects' time history (including self)
							Survey Markers
						Assumptions (GPS objects?)
					Links
						Relative Observations
						Motion Estimates
							IMU data
							Kalman style interpolation
	Current implementation
		Strengths
			Encodes appropriate information in a covariance matrix form
			Cleanly integrates multiple observations into an object model
		Weaknesses
			Camera model does not account for uncertainty from angle of pixels, gimbal pose, IMU sample time (angular velocity), GPS drift
			GPS drift is deliberately ignored in favour of applying a generous velocity uncertainty to the objects.
			Cannot describe complex distributions (unbounded polynomial order)
				arbitrary geometry Uncertainty Images from SoggyDog
	Future Work
			Velocity interpolation not implemented.
			Graph traversal (SLAM) not implemented.
			Describe hyperbolic distribution and flaws
			Describe Laser beam necking case


\section{Test Results}

	Software tests
		Gimbal pose, IMU data, GPS location etc
		Camera uncertainty models combined with assumptions
		Pretty pictures
			Multiple observations with time-evolution

	Live tests
		*Able to physically follow one object, while tracking multiple others.
		Acceleration limits, velocity limits, time-cutoffs.
		pitch and roll stability
		Basic colour matching limitations

\section{Future Work}
	Expand uncertainty analysis
	SLAM, end to end solution
	SIFT algorithm
	Optical Flow
	Recommendations:
		ROS, ROS, ROS.

\section{Conclusions}
	We have greatly improved the capabilities of the UWA autonomous hexacopter platform, and have brought the code-base up to a level where it is feasible to implement a SLAM process.





%Referencing
%---------------------------------------------------------
%\pagebreak
\renewcommand{\refname}{References}
\addcontentsline{toc}{section}{References}
\bibliography{references/refs}
\bibliographystyle{ieeetr}
%\addbibresource{references/refs.bib}
%\printbibliography

%---------------------------------------------------------

\begin{appendices}
  %\addcontentsline{toc}{section}{Appendices}
  \section{ArduPilot (Pixhawk) Proposal} \label{sec:PihawkProposal}
    \includepdf[pages=-]{../picopter/Documents2015/ardupilot-proposal/ardupilot-proposal.pdf}

\end{appendices}

	
\end{document}

